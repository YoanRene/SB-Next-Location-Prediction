\begin{opinion}

    El trabajo de diploma \textbf{Predicción de la siguiente ubicación utilizando contexto
    basado en sentimientos}, presentado por el estudiante \textbf{Yoan René Ramos
    Corrales}, para optar por el título de licenciado en Ciencia de la Computación, se
    corresponde con intereses del grupo de Modelación Biomatemática de la facultad
    de Matemática y Computación con el objetivo de explorar y proponer nuevas
    estrategias en el modelado, solución y manejo de problemas aplicados a las
    biociencias.

    Cuando hice propuestas para temas de diploma, entre mis aspiraciones estaba
    obtener una segunda versión de una \textbf{Simulación computacional para la
    dinámica de enfermedades transmitidas por vectores}, defendida el año
    anterior por el ya licenciado y tutor de esta tesis Ernesto Alfonso Hernández, pero
    la creatividad de Yoan René me llevó por otros derroteros, aunque la relación
    entre ellos, permite augurar provecho para investigaciones futuras y el
    perfeccionamiento de ambas metodologías.

    Desde un inicio, Yoan René fue muy independiente y resuelto, aunque manejó
    varias opciones antes de la decisión final. Era como si hubiera construido en su
    cerebro la heurística que seguiría para lograr sus objetivos.

    Para desarrollar este trabajo de tesis, el diplomante tuvo que realizar una amplia
    investigación en temas que no forman parte del currículo de pregrado y que
    además muchas resultan muy actuales y novedosas y se renuevan con mucha
    rapidez. Los pasos para lograr los objetivos específicos, incluyeron la formulación
    de un modelo para predecir la siguiente ubicación de un individuo a partir de la
    combinación del contexto espacio-temporal considerando algunos sentimientos
    como variable de estudio para mejorar la precisión de la predicción. Para ello,
    propone un modelo de aprendizaje profundo lo que le permitió considerar
    dependencias entre variables y determinar cómo los estados de un individuo
    afectan la elección de su siguiente ubicación.

    Debo confesar que a mí, el título no me gusta mucho, quizás por lo evidente, pero
    el trabajo es de Yoan René, con una importante asesoría de Ernesto y se merece
    el privilegio de esa elección por la independencia y creatividad con que ha
    trabajado.

    Yo soy una profesora dichosa porque he tenido la dicha en muchas ocasiones de
    que me hayan elegido como tutora (debería ser más como consultante)
    estudiantes con muy buena preparación, muy buenas ideas y muchos deseos de
    presentar trabajos de tesis interesantes y útiles. Y Yoan René, lo confirma.

    Como afirma el diplomante en las Conclusiones de su trabajo, esta investigación
    además, ha comparado el modelo propuesto que incorpora el contexto de
    sentimientos inferidos mediante un modelo de lenguaje analizando su desempeño
    comparándolo con diferentes modelos para la predicción de la próxima ubicación
    de usuarios usando conjunto de datos Geolife y el modelo basado en MHSA
    Transformer. Y aunque los resultados mostraron que la inclusión del contexto de
    sentimientos no reportó mejoras estadísticamente significativas, su potencial
    utilidad podría depender de la calidad de la inferencia emocional y del tipo de
    movilidad analizada y desde ya propone en sus Recomendaciones, entre otras
    investigaciones futuras, el uso de fuentes de datos adicionales, como interacciones
    en redes sociales, encuestas directas o datos biométricos, que puedan
    proporcionar una inferencia más precisa y menos ruidosa de los sentimientos, para
    manejar mejor el estado emocional de las personas y esto nos esperanza por la
    repercusión que tendrían estos estudios en la predicción de enfermedades
    contagiosas.

    Muy parecido a como me pasó con su tutor, me sorprendió favorablemente la
    buena calidad ortográfica y de redacción del documento presentado por Yoan
    René.

    En un tiempo limitado y superando las dificultades de un contexto como el de
    nuestro país, máxime en las condiciones actuales de carencias energéticas,
    representa un reto significativo que el diplomante lograra presentar un trabajo de
    una calidad indiscutible. De seguro hay mucho que perfeccionar, pero estamos
    satisfechos con su desempeño, aseguramos que cumple con los requerimientos
    para optar por el título de licenciado considerando que merece la calificación de
    Excelente.\\

    Muchas Felicidades!!!
    
    
\begin{figure}[h]
    \hspace{1.5cm}
    \vspace{-0.5cm}
    \includegraphics[width=0.125\textwidth ]{Graphics/firma}   
\end{figure}

\makebox[5cm]{\hrulefill} 

Dra. Aymée Marrero Severo

Tutora
\end{opinion}