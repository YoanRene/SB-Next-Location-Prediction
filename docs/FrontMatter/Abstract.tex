% Abstract.tex
\begin{resumen}
	La predicción de la siguiente ubicación de un individuo 
	es crucial para 
	analizar el comportamiento de la movilidad humana. 
	Los modelos existentes que indagan en esta problemática a 
	menudo suelen ignorar información de contexto. Por otra parte, 
	el análisis de sentimientos es una rama de estudio que se 
	ha desarrollado en los últimos años, extrayendo información 
	de redes sociales para predecir el estado emocional de sus usuarios.
	En este trabajo se propone un modelo que busca predecir la 
	siguiente ubicación a partir de la combinación del contexto 
	espacio-temporal con variables de sentimientos, en pos de mejorar 
	la precisión de la predicción. Para ello, se propone diseñar un 
	modelo de aprendizaje profundo que permita extraer las 
	dependencias complejas entre variables y determinar cómo los 
	estados de un individuo afectan la elección de su siguiente ubicación.\\
	%Palabras clave

	\textbf{Palabras Clave:}
	predicci\'on de la siguiente ubicación, movilidad humana, sentimientos, aprendizaje profundo 
\end{resumen}
	
\begin{abstract}
	Predicting an individual's next location is crucial for analyzing 
	human mobility behavior. Existing models that address this issue often 
	tend to ignore contextual information. On the other hand, sentiment 
	analysis is a field of study that has developed in recent years, 
	extracting information from social networks to predict the emotional 
	state of users. In this work, a model is proposed that aims to 
	predict the next location by combining spatio-temporal context with 
	sentiment variables, with the goal of improving prediction accuracy.
	To achieve this, the proposal is to design a deep learning model 
	capable of extracting complex dependencies between variables and 
	determining how an individual's emotional states influence the 
	choice of their next location.\\

	\textbf{Keywords:}
	next location prediction, human mobility, sentiment analysis, deep learning
\end{abstract}