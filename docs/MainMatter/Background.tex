\chapter{Estado del Arte}\label{chapter:state-of-the-art}
El problema de predicción de la siguiente ubicación de una persona ha encontrado 
aplicación en diversos campos: como sistemas de recomendación (\cite{xue2021mobtcast}), 
redes de sensores (\cite{pirozmand2014human}) y análisis del comportamiento de movilidad~(\cite{wang2021reinforced,xu2022understanding}). 

La definición exacta del problema varía entre los estudios dependiendo de 
los diferentes objetivos y conjuntos de datos empleados. Por ejemplo, 
las aplicaciones de redes sociales basadas en la ubicación (LBSN\footnote{LBSN: Location-Based Social Networks}) se 
centran en predecir el siguiente punto de interés (POI\footnote{POI: Point of Interest}) 
registrado (\cite{wang2022online,xue2021mobtcast}). En contraste, los estudios del comportamiento 
de movilidad buscan comprender la siguiente ubicación de un usuario para 
realizar una actividad~(\cite{solomon2021analyzing}). Aquí, la investigación est\'a centrada en los métodos 
propuestos para aplicaciones de movilidad.

\section{Modelos de Markov}
La última década ha presenciado la expansión de estudios centrados en la 
predicción de la siguiente ubicación. La Cadena de Markov y sus variantes 
son probablemente los métodos más empleados para esta tarea~(\cite{luca2021survey}). 
Estos modelos consideran las ubicaciones como estados y construyen una 
matriz de transición que codifica la probabilidad de transición entre 
estados para cada individuo. 

\cite{ashbrook2002learning} y \cite{gambs2012next} propusieron identificar ubicaciones significativas 
 a partir de datos GPS y construir un modelo de Markov para predecir 
 las transiciones de ubicación. 
 
 Posteriormente, las variantes del 
 modelo de Markov que consideran movimientos colectivos~(\cite{chen2014nlpm}) e 
 incorporan la importancia de la ubicación~(\cite{huang2017mining}) aumentaron aún 
 más el rendimiento de la predicción. Sin embargo, los modelos basados
en Markov tienen dificultades para representar los complejos patrones
secuenciales en la movilidad humana debido a su supuesto inherente 
de que el estado actual solo depende de los estados de pasos de tiempo
 previamente limitados~(\cite{li2020hierarchical}).

\section{Modelos de Aprendizaje Profundo}
Los avances recientes en el aprendizaje profundo (DL\footnote{DL: Deep Learning}) también han 
promovido su aplicación en la predicción de ubicación. Como método 
de modelado de secuencias ampliamente adoptado, se ha informado que 
los modelos basados en redes neuronales recurrentes (RNN\footnote{RNN: Recurrent Neural Networks}), como es el caso del
modelo de memoria a corto y largo plazo (LSTM\footnote{LSTM: Large Short-Term Memory})~(\cite{solomon2021analyzing}) y la RNN 
espacio-temporal (ST\footnote{ST: Spatial-Temporal})~(\cite{liu2016predicting}), superan a los modelos de Markov
 por un amplio margen en esta tarea. 
 
 Aún así, los modelos RNN 
 convencionales tienden a subestimar las dependencias a largo 
 plazo cuando aumenta la longitud de la secuencia de entrada. 
 Por lo tanto, los estudios emplearon un mecanismo de atención 
 para capturar dinámicamente las dependencias tanto a corto 
 como a largo plazo~(\cite{feng2018deepmove,li2020hierarchical}). 
 
 Además, el modelo Transformer, 
 que se basa en el mecanismo de autoatención multicabezal (MHSA\footnote{MHSA: Multi-Head Self-Attentional})~(\cite{vaswani2017attention}), 
 ha comenzado a ganar interés en el campo. En particular, \cite{xue2021mobtcast}
 propusieron MobTcast para considerar 
  varios contextos con una estructura basada en Transformer 
  y obtuvieron resultados de vanguardia en la predicción de POI 
  para datos LBSN. 
  
  A pesar de tener un gran potencial para 
  aprender las complejas dependencias espacio-temporales, pocos 
  estudios han aplicado Transformer al problema de predicción 
  de ubicación debido al acercamiento reciente de estos modelos a la problem\'atica en cuesti\'on.

\section{Desaf\'ios del Problema}
Comprender los factores que afectan la elección de la ubicación de 
la actividad es beneficioso para predecir la movilidad de las personas, 
ya que pueden considerarse como conocimiento previo y potencialmente 
guiar el aprendizaje de los modelos DL. En el entorno de la movilidad, 
la elección de ubicaciones se considera una parte integral
del comportamiento de actividad-viaje de las personas y se ha
estudiado dentro del marco basado en la actividad~(\cite{schonfelder2016urban}). 

Los estudios centrados en analizar el comportamiento de 
viaje a lo largo del tiempo sugieren que se encuentra tanto 
estabilidad como variabilidad en las elecciones de ubicación 
de actividad de las personas. Por ejemplo, \cite{dharmowijoyo2017analysing} mostraron que la variabilidad de las visitas
a ubicaciones es mucho mayor entre los pares fin de 
semana-día laborable que entre los pares día laborable-día 
laborable y fin de semana-fin de semana. 

Los estudios 
empíricos también demuestran la correlación de diferentes
aspectos del comportamiento de viaje individual. Por ejemplo, \cite{susilo2014repetitions} informaron una alta repetición 
en las combinaciones de ubicación-modo de viaje, lo que sugiere que las 
personas usan el mismo modo de viaje para llegar a sus ubicaciones. 
\cite{hong2022conserved} llegaron a conclusiones similares, donde 
encontraron que solo un subconjunto de todas las combinaciones de 
ubicación-modo es esencial para describir el comportamiento de movilidad. 
Desde esta perspectiva, los aspectos del comportamiento de viaje 
pueden considerarse restricciones para la elección de ubicaciones de 
actividad de las personas.

Un problema similar a la predicción de la siguiente ubicación es la 
formulación del conjunto de opciones de ubicación de un individuo, que 
es un componente crucial en los modelos de microsimulación de 
tráfico~(\cite{mariante2018modeling}). En lugar de predecir la siguiente ubicación exacta, 
el problema se centra en generar un conjunto que contenga todas las 
ubicaciones posibles. Basándose en la teoría de la geografía del tiempo, 
se ha aplicado el análisis de áreas de rutas potenciales para abordar 
el problema, lo que sugiere que el conjunto de opciones está limitado 
por el tiempo de viaje~(\cite{scott2012modeling}), la hora del día~(\cite{yoon2012feasibility}) y el modo 
de viaje disponible~(\cite{neutens2007space}). Sin embargo, este conocimiento empírico 
no se utiliza plenamente en los modelos para predicciones de ubicación.

La mayoría de los estudios existentes solo utilizan secuencias de 
visitas de ubicaciones por los individuos para predecir sus 
próximas ubicaciones, independientemente del contexto de movilidad 
actual (\cite{laha2018real}). Esto se debe a la complejidad y 
diversidad de los datos de contexto (\cite{tedjopurnomo2020survey}). 

Con el desarrollo del aprendizaje automático, los estudios han 
comenzado a explorar métodos para la fusión de datos de contexto y 
emplearlos para impulsar la investigación de la movilidad humana 
(\cite{zheng2018survey,lau2019survey}). Sin embargo, los datos de 
contexto de múltiples fuentes poseen formatos complejos, lo que 
dificulta su fusión en una representación unificada (\cite{liao2018multi}). 

Los marcos de DL sensibles al contexto proporcionan 
una solución para combinar información de contexto de múltiples fuentes 
(\cite{sun2022tcsanet}), pero con mayor frecuencia se centran en el aspecto 
temporal y pasan por alto las interacciones espacio-temporales complejas.\\

Mencionar que la evoluci\'on de la problem\'atica y el auge de los grandes modelos de lenguaje natural (LLM\footnote{LLM: Large Language Models})
han puesto a investigar otras formas de atacar el asunto en cuesti\'on con muy buenos resultados (\cite{wang2024inextlargelanguage}).
De aqu\'i y del tema de la utilizaci\'on del contexto para mejorar la predicci\'on, que se mantenga la necesidad de explorar
m\'as a fondo las influencias de la informaci\'on de contexto del individuo. 

En particular, se har\'a
enf\'asis en el uso de sentimientos
o emociones que puedan influir en la elecci\'on de la siguiente ubicaci\'on. 
Dichos sentimientos, que en un escenario ideal podr\'ian ser extra\'idos de las redes sociales, 
por falta de datos p\'ublicos se hace casi inviable. Entonces ser\'ia pr\'actico el empleo de los ya 
mencionados modelos del lenguaje natural para inferir un sentimiento 
a partir de los datos que se tienen, como lo son la ubicaci\'on y el momento actual del individuo.
Luego, queda hacer uso de un modelo que 
tenga en cuenta dichos sentimientos para mejorar la predicci\'on de la siguiente ubicaci\'on.

Se propone entonces utilizar el modelo de Aprendizaje Profundo basado en Transformers y MHSA de \cite{Hong_2023} 
por sus excelentes resultados en esta problem\'atica y su capacidad 
de capturar dependencias.