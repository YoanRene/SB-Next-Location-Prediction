% Introduction.tex
\chapter*{Introducción}\label{chapter:introduction}
\addcontentsline{toc}{chapter}{Introducción}

Las enfermedades transmitidas por vectores son una de 
las grandes problemáticas de salud pública en el mundo.
Según la Organización Mundial de la Salud (OMS),
estas representan más del 17\% de las enfermedades infecciosas
y provocan más de 700,000 desfunciones anuales (\cite{who_website2024}),
lo que supone un gran impacto en la salud y el bienestar
de las poblaciones afectadas .
\par
Estas enfermedades son causadas por parásitos,
bacterias o virus que son transmitidos por insectos,
garrapatas o ácaros, y pueden causar una amplia variedad
de síntomas, desde leves hasta graves y potencialmente
mortales. As\'i como en el impacto directo en la salud de
las personas, estas enfermedades también tienen un
impacto económico significativo en los países afectados,
ya que pueden dañar la productividad laboral,
el turismo y otros sectores económicos.
\par
En este contexto, la prevención y el control de las
enfermedades transmitidas por vectores son esenciales
para proteger la salud pública y reducir su impacto
en la sociedad. Esto implica una amplia gama de
estrategias, desde la educación y la concientización
de la población sobre las medidas de prevención,
hasta la implementación de programas de control
de vectores y la investigación y desarrollo de nuevos
medicamentos y vacunas.
\par
La din\'amica de estas enferemedades es compleja y
multifactorial, a\'un m\'as cuando es transmitida r\'apidamente en el tiempo, por lo que
para ser analizada se requiere de un enfoque interdisciplinario que
combine conocimientos de epidemiología,
biología, matemáticas, computaci\'on y otras disciplinas.
En este sentido, los modelos matemáticos y computacionales 
pueden ser una herramienta valiosa para entender mejor
el comportamiento de estas enfermedades y predecir su
evolución.

%\section{Motivaci\'on}

\section{Motivaci\'on}

La aparición de una epidemia de este tipo de enfermedades
presenta desafíos complejos que demandan una respuesta 
inmediata y coordinada.  Desde la devastadora 
pérdida de vidas humanas hasta las profundas 
consecuencias económicas y sociales, la necesidad de 
herramientas eficaces para la gestión de estas 
situaciones es innegable.

El colapso de los sistemas de salud debido a la rápida 
propagación de enfermedades, a menudo exacerbado 
por una capacidad limitada de respuesta, subraya 
la importancia de la preparación y la prevención.  
La escasez de recursos, la sobrecarga del personal 
médico y la dificultad para brindar atención 
adecuada a todos los necesitados son consecuencias 
directas de una respuesta inadecuada ante una 
epidemia.

As\'i como las las epidemias imponen una carga 
económica significativa, las medidas de control, 
como cuarentenas y restricciones de viaje, junto 
con el aumento de los gastos médicos, pueden 
desencadenar crisis económicas.  La disminución de 
la productividad, la interrupción del turismo y 
las pérdidas en diversos sectores económicos son 
elementos que agravan aún más la situación.

Ante estos desafíos, el desarrollo de una herramienta 
computacional de simulación epidémica se presenta 
como una solución crucial. Esta herramienta permitiría 
modelar la dinámica de este tipo de enfermedades, 
proporcionando información vital para la toma de 
decisiones, fortaleciendo as\'i la capacidad de 
respuesta ante futuras 
epidemias, mitigando su impacto y protegiendo la 
salud pública.

%\section{Antecedentes}

\section{Antecedentes}
La experiencia del Grupo de Biomatemática de la 
Facultad de Matemática y Computación en el trabajo 
con modelos epidemiológicos poblacionales, incluyendo 
la estimación y ajuste de sus parámetros, respalda 
la necesidad de herramientas computacionales que 
aborden los desafíos inherentes a la modelación, 
solución, estimación y predicción en epidemiología. 
En este contexto, se destaca la tesis de licenciatura 
``Redes Complejas en Epidemiología. Aplicaciones a 
modelos de VIH y Dengue'' de Glenda Beatriz Rodríguez 
García (2016), as\'i como ``Simulación computacional para la
dinámica de enfermedades transmitidas
por vectores'' de Ernesto Alfonso Hern\'andez (2024), ambos bajo
la tutor\'ia de la Dra. Aymée Marrero Severo. Estas investigaciones 
introducen conceptos 
de la teoría de redes complejas y analizan su 
aplicación en la modelación de epidemias como el  
dengue. Estos trabajos previos han demostrado la 
eficacia del uso de redes complejas en la comprensión 
y predicción de la dinámica de epidemias,
sirviendo como justificación para el 
desarrollo de la herramienta computacional propuesta.

%Problema de Investigaci\'on
\section{Problema de Investigaci\'on}
Para comprender y modelar la dinámica de transmisión 
de enfermedades por vectores, la simulación 
computacional se presenta como una herramienta 
fundamental. Permite explorar diversos escenarios 
y estrategias de control, con el objetivo de 
prevenir y mitigar la propagación de enfermedades 
(\cite{ferguson2006strategies}). Diversos modelos contribuyen 
al estudio de esta dinámica, desde el clásico modelo 
SIR (\cite{kermack1927contribution}), que clasifica a la 
población en susceptibles, infectados y recuperados, 
hasta modelos basados en agentes, que simulan el 
comportamiento individual, y modelos de redes, que 
representan las interacciones entre vectores, 
huéspedes y entorno (\cite{ferguson2006strategies}; \cite{balcan2009multiscale}). Cada método ofrece diferentes 
recursos para el proceso de simulación. En este 
contexto, la presente investigación se centra 
en la modelación de una simulación para la 
dinámica de enfermedades transmitidas por 
vectores, con énfasis en la incorporación de 
elementos que simulen la toma de decisiones 
humanas.

%Pregunta Cient\'ifica
\section{Pregunta Cient\'ifica}
¿Es posible implementar una herramienta de simulación, la cual modele
la propagación de una enfermedad transmitida por vectores teniendo en cuenta el
comportamiento dinámico de las personas?

%Objetivos
\section{Objetivos}
\subsection{Objetivos General}
Desarrollar un modelo de aprendizaje autom\'atico capaz de generar redes complejas para la simulación de la propagación de una enfermedad transmitida por vectores teniendo en cuenta el comportamiento dinámico de las personas.

\subsection{Objetivos Espec\'ificos}
\begin{itemize}
    \item Implementar un modelo de aprendizaje autom\'atico para la generación de redes complejas.
    \item Implementar un modelo de simulación para la propagación de una enfermedad transmitida por vectores.
    \item Implementar un modelo de comportamiento dinámico de las personas.  
\end{itemize}

%Estructura
\section{Estructura}
La estructura del documento est\'a dada por tres capítulos. En el Capítulo 1 se
realiza un estudio sobre el marco teórico conceptual del problema. El Capítulo 2 aborda tanto la modelación como la metodolog\'ia utilizada. 
En el Capítulo 3
se discuten sobre la implementación del modelo, as\'i como de su evaluaci\'on mediante la experimentaci\'on. Terminado este \'ultimo cap\'itulo
se da paso a las Conclusiones, la cual resume los puntos principales,
dando respuesta a los objetivos según los resultados obtenidos.