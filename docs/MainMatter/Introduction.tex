% Introduction.tex
\chapter*{Introducción}\label{chapter:introduction}
\addcontentsline{toc}{chapter}{Introducción}

El rápido proceso de urbanización ha incrementado los viajes 
individuales, presentando desafíos para la sostenibilidad de las 
ciudades. Para alcanzar los objetivos de desarrollo de las 
Naciones Unidas (\cite{Griggs2013}), los cambios en el comportamiento de 
la movilidad y los nuevos conceptos para promover estos, 
desempeñarán un papel fundamental
 (\cite{Martin2021}). En la salud p\'ublica, la pandemia de la COVID-19 es otra prueba
de la necesidad de conocer esto a fondo, puesto
que la movilidad es un factor determinante en la propagación
de enfermedades infecciosas  (\cite{Kraemer2020}). Definiciones recientes como
Movilidad como
Servicio (MaaS) (\cite{Reck2022}), carga inteligente (\cite{Xu2018}) y 
viaje compartido (\cite{Huang2019}), dependen de la capacidad de ofrecer 
servicios personalizados adaptados al contexto del viaje y las 
características individuales (\cite{Ma2022}).
La movilidad individual, es decir, predecir cuándo y dónde ocurrirá 
un viaje, es esencial para el desarrollo de estos conceptos y, 
por lo tanto, una herramienta clave para el transporte sostenible.

\section{Motivación}
La predicción de la siguiente ubicación de un individuo, a partir 
de su información histórica de movilidad, es un problema que ha cobrado 
gran relevancia en la última década. El auge de los modelos de 
aprendizaje profundo (\cite{luca2021survey}) ha impulsado el interés de 
los investigadores en abordar este problema con métodos basados 
en el aprendizaje.  Al formularse como un problema de predicción 
de secuencias, similar a tareas en procesamiento del lenguaje 
natural y de audio, se aplican modelos exitosos en estos campos.  
En particular, el modelo Transformer (\cite{vaswani2017attention}), con su mecanismo 
de autoatención multicabezal, ha revolucionado el modelado de 
secuencias.
\newpage
\section{Antecedentes}

Los modelos Transformer se están utilizando en la predicción de la 
movilidad individual por su capacidad para abordar desafíos 
específicos: 

(1) Capturar las múltiples periodicidades en los 
patrones de visitas a ubicaciones, las cuales varían entre 
individuos (\cite{feng2018deepmove}). 

(2) Modelar la dependencia a largo plazo del 
comportamiento de movilidad, considerando que la movilidad actual 
se ve influenciada por comportamientos previos (\cite{Cherchi2017,Sun2013}).

\section{Problema de Investigación}

La movilidad humana presenta características únicas, como 
dependencias espacio-temporales complejas (\cite{feng2018deepmove,li2020hierarchical}) y la 
estocasticidad de las visitas a ubicaciones (\cite{Song2010}), que 
dificultan la aplicación directa de modelos de aprendizaje de 
secuencias. Predecir la siguiente ubicación a partir del 
historial de visitas es un desafío. Un modelo preciso debe 
considerar el contexto que influye en la elección de ubicaciones.  
Estudios de comportamiento de viaje sugieren que la selección de 
ubicaciones se correlaciona con aspectos como la disponibilidad 
de modos de transporte (\cite{Hong_2022}) y el día de la semana (\cite{Dharmowijoyo2016}). 
¿Pero qu\'e otros factores influir\'ian en esto? ¿Ser\'ia posible
que los sentimientos o estados de un individuo afecten la elección
de su siguiente ubicación? Por ejemplo, si un individuo
está de mal humor, puede ser que vaya a un lugar
de entretenimiento, como un cine o un bar; mientras que si
se encuentra en un estado de enfermedad es probable que
vaya a un hospital o a una farmacia. De aqu\'i surge la
problem\'atica de estudio:\\

Predicci\'on de la siguiente ubicación de un individuo utilizando contexto
basado en los sentimientos o estados de este.

\section{Pregunta Científica}

¿Cómo influyen los sentimientos o estados de un indivudo en 
la predicción de su siguiente 
ubicación, y cómo se puede integrar esta 
información en un modelo computacional para mejorar la 
precisión de la predicción?

\section{Objetivos}
\subsection{Objetivo General}

Desarrollar un modelo de aprendizaje profundo para la predicción de la siguiente ubicación que 
incorpore la información de sentimiento o estado del individuo, 
incluyendo el historial de ubicaciones, y datos 
temporales.

\subsection{Objetivos Específicos}

\begin{itemize}
    \item Representar el sentimiento a partir de la información 
    que se tiene del conjunto de datos.
    \item Diseñar un modelo de redes neuronales basado en Transformer que utilice 
    información de ubicación, el sentimiento o estado, y tiempo, para predecir 
    la siguiente ubicación.
    \item Evaluar la mejora en el rendimiento de la predicción al 
    utilizar el contexto del sentimiento.
\end{itemize}

%Estructura
\section{Estructura}
La tesis est\'a estructurada en tres capítulos. En el 
Capítulo 1 se
realiza un estudio sobre el marco teórico conceptual del problema. 
El Capítulo 2 aborda tanto la modelación como la metodolog\'ia utilizada. 
En el Capítulo 3
se discuten sobre la implementación del modelo, as\'i como de su 
evaluaci\'on mediante la experimentaci\'on. Finalmente se presentan las Conclusiones, donde se resumen los aspectos 
fundamentales y se analiza el cumplimiento de los objetivos propuestos; y las Recomendaciones donde se formulan los trabajos futuros.\\
