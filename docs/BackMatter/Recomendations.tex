\begin{recomendations}
    A partir de los resultados, se proponen las siguientes 
    recomendaciones para futuras investigaciones y aplicaciones:

\begin{itemize}
    \item \textbf{Evaluación rigurosa del LLM:} Si se emplea un LLM para 
    la generación de etiquetas de sentimientos, es fundamental realizar 
    una evaluación exhaustiva de su desempeño. Esto incluye verificar 
    la coherencia y estabilidad de las inferencias, comparar los 
    resultados con anotaciones humanas y analizar posibles sesgos o 
    errores sistemáticos que puedan afectar la calidad de los datos.
    
    \item \textbf{Uso de técnicas clásicas de recuperación de información:} 
    En lugar de utilizar un 
    LLM, una alternativa viable es aplicar enfoques tradicionales de 
    recuperación de información, como la búsqueda de patrones en 
    bases de datos preexistentes de estados de la persona, el uso de 
    diccionarios léxicos de sentimientos o modelos supervisados de 
    clasificación de texto. 
    Estos métodos pueden ofrecer mayor control y transparencia en 
    la inferencia de sentimientos, reduciendo la variabilidad 
    introducida por la generación de texto de los LLM.

   

    \item \textbf{Mejora en la inferencia del contexto de sentimientos:} 
    Para capturar mejor el estado emocional de los usuarios, se recomienda 
    el uso de fuentes de datos adicionales, como interacciones en redes 
    sociales, encuestas directas o datos biométricos, que puedan 
    proporcionar una inferencia más precisa y menos ruidosa de los 
    sentimientos.
    
    \item \textbf{Análisis del impacto del contexto emocional en 
    diferentes escenarios:} Se sugiere evaluar el impacto del contexto 
    de sentimientos en distintos conjuntos de datos de movilidad, 
    especialmente en entornos donde las decisiones de movimiento pueden 
    estar más influenciadas por factores emocionales.
    
    \item \textbf{Validación en contextos reales:} Para evaluar la 
    aplicabilidad práctica de los modelos, se recomienda realizar 
    pruebas en escenarios reales con usuarios que proporcionen 
    retroalimentación sobre la precisión de las predicciones y 
    la relevancia del contexto emocional en su movilidad cotidiana.
\end{itemize}

En general, estos puntos pueden contribuir a mejorar la precisión de 
los modelos de predicción de movilidad y a profundizar en la 
comprensión del papel de los factores de sentimientos en los patrones 
de movimiento de los usuarios.

\end{recomendations}
