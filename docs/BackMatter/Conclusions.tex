\begin{conclusions}
    En este trabajo, se analiza el desempeño de diferentes modelos para 
    la predicción de la próxima ubicación de usuarios en el conjunto de 
    datos Geolife. Se comparó el modelo basado en MHSA Transformer con 
    nuestro modelo propuesto, el cual incorpora el contexto de sentimientos 
    inferidos mediante un modelo de lenguaje.

Los resultados muestran que ambos modelos basados en aprendizaje profundo 
superan significativamente al método de primer orden 1-MMC en casi todos 
los indicadores de desempeño, resaltando la importancia de modelar 
dependencias temporales a largo plazo en tareas de predicción de 
movilidad. Sin embargo, la inclusión del contexto de sentimientos en 
nuestro modelo no resultó en mejoras estadísticamente significativas en 
comparación con el MHSA Transformer, según la prueba U de Mann-Whitney.

Se identificaron dos posibles razones para esta falta de significancia: 
(i) la inferencia del contexto de sentimientos basada en información 
limitada (hora, lugar y día de la semana) pudo no haber capturado con 
precisión el estado emocional real de los usuarios, introduciendo ruido 
en los datos, y (ii) la influencia de los sentimientos en la movilidad
puede ser marginal cuando los patrones de movimiento están fuertemente 
determinados por factores estructurales, como el trabajo o la rutina diaria.

Por otro lado, el análisis del impacto de la longitud de la secuencia 
histórica reveló que la mejor configuración para el modelo propuesto 
se obtiene al considerar las secuencias de los últimos 7 días. 
Además, los picos en el desempeño observados en los días 7 y 14 sugieren 
que los patrones de movilidad pueden tener periodicidades semanales 
relevantes para la predicción.

En conclusión, si bien el contexto de sentimientos no proporcionó mejoras 
significativas en la predicción de movilidad en este estudio, su 
potencial utilidad podría depender de la calidad de la inferencia 
emocional y del tipo de movilidad analizada.

\end{conclusions}
