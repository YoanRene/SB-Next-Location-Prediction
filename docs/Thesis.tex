\documentclass[12pt,oneside]{uhthesis}
\usepackage{subfigure}
\usepackage[ruled,lined,linesnumbered,titlenumbered,algochapter,spanish,onelanguage]{algorithm2e}
\usepackage{amsmath}
\usepackage{amssymb}
\usepackage{amsbsy}
\usepackage{caption,booktabs}
\captionsetup{ justification = centering }
%\usepackage{mathpazo}
\usepackage{float}
\setlength{\marginparwidth}{2cm}
\usepackage{todonotes}
\usepackage{listings}
\usepackage{xcolor}
\usepackage{multicol}
\usepackage{graphicx}
\floatstyle{plaintop}
\restylefloat{table}
\addbibresource{Bibliography.bib}
% \setlength{\parskip}{\baselineskip}%
\renewcommand{\tablename}{Tabla}
\renewcommand{\listalgorithmcfname}{Índice de Algoritmos}
%\dontprintsemicolon
\SetAlgoNoEnd

\definecolor{codegreen}{rgb}{0,0.6,0}
\definecolor{codegray}{rgb}{0.5,0.5,0.5}
\definecolor{codepurple}{rgb}{0.58,0,0.82}
\definecolor{backcolour}{rgb}{0.95,0.95,0.92}

\lstdefinestyle{mystyle}{
    backgroundcolor=\color{backcolour},   
    commentstyle=\color{codegreen},
    keywordstyle=\color{purple},
    numberstyle=\tiny\color{codegray},
    stringstyle=\color{codepurple},
    basicstyle=\ttfamily\footnotesize,
    breakatwhitespace=false,         
    breaklines=true,                 
    captionpos=b,                    
    keepspaces=true,                 
    numbers=left,                    
    numbersep=5pt,                  
    showspaces=false,                
    showstringspaces=false,
    showtabs=false,                  
    tabsize=4
}

\lstset{style=mystyle}

\title{Título de la tesis}
\author{\\\vspace{0.25cm}Nombre del autor}
\advisor{\\\vspace{0.25cm}Nombre del primer tutor\\\vspace{0.2cm}Nombre del segundo tutor}
\degree{Licenciado en (Matemática o Ciencia de la Computación)}
\faculty{Facultad de Matemática y Computación}
\date{Fecha\\\vspace{0.25cm}\href{https://github.com/username/repo}{github.com/username/repo}}
\logo{Graphics/uhlogo}
\makenomenclature

\renewcommand{\vec}[1]{\boldsymbol{#1}}
\newcommand{\diff}[1]{\ensuremath{\mathrm{d}#1}}
\newcommand{\me}[1]{\mathrm{e}^{#1}}
\newcommand{\pf}{\mathfrak{p}}
\newcommand{\qf}{\mathfrak{q}}
%\newcommand{\kf}{\mathfrak{k}}
\newcommand{\kt}{\mathtt{k}}
\newcommand{\mf}{\mathfrak{m}}
\newcommand{\hf}{\mathfrak{h}}
\newcommand{\fac}{\mathrm{fac}}
\newcommand{\maxx}[1]{\max\left\{ #1 \right\} }
\newcommand{\minn}[1]{\min\left\{ #1 \right\} }
\newcommand{\lldpcf}{1.25}
\newcommand{\nnorm}[1]{\left\lvert #1 \right\rvert }
\renewcommand{\lstlistingname}{Ejemplo de código}
\renewcommand{\lstlistlistingname}{Ejemplos de código}

\begin{document}

\frontmatter
\maketitle

\begin{dedication}
    A todos los que me ayudaron \\a construir el camino\\ hasta aquí
\end{dedication}
\begin{acknowledgements}
    Agradezco a mi tutora, Aymée, por sus correcciones y sugerencias, 
    y a mi cotutor, Ernesto, por sus ideas y guía durante el desarrollo 
    de este trabajo. También agradezco a mi fiel lectora, la 
    periodista Claudia, por sus comentarios y observaciones.

    A mis profesores, por su dedicación, sus exigencias y enseñanzas,
     que no solo han sido fundamentales en mi formación académica, 
     sino que también han inspirado mi crecimiento personal y 
     profesional. 

Extiendo mi gratitud a mi familia por su apoyo incondicional, en 
especial a mi tía Liset, a mi mamá y a mi abuela, por su paciencia y 
comprensión. A mis amigos de la universidad, desde el primer año, 
en especial a Emmanuel, el Viti y Rubén, hasta el último año, a 
Roger Moreno, Marquito Puente 100, Kevin C122 y Juana Carla, 
por los momentos compartidos.

A mi compañera de estudios, Claudia, por transitar juntos este camino 
de superación y aprendizaje. 

Finalmente, agradezco a todos aquellos que 
no se mencionan, pero que de alguna manera han contribuido a que esta 
tesis sea posible.

\end{acknowledgements}
\begin{opinion}
    Opiniones de los tutores
\end{opinion}
% Abstract.tex
\begin{resumen}
	El presente trabajo se centra en el desarrollo de un modelo computacional para simular la movilidad humana y su impacto en la dinámica de enfermedades transmitidas por vectores. La movilidad humana, caracterizada por los desplazamientos de individuos entre diferentes ubicaciones geográficas, juega un papel crucial en la propagación de estas enfermedades.  El modelo propuesto incorpora diversos factores que influyen en la movilidad, como patrones de desplazamiento diarios, viajes de larga distancia y migraciones, para comprender cómo estos movimientos afectan la distribución espacial y temporal de las enfermedades.  Se espera que este modelo contribuya al diseño de estrategias de control más efectivas para mitigar el impacto de estas enfermedades en la salud pública.
	\end{resumen}
	
	\begin{abstract}
	This thesis focuses on the development of a computational model to simulate human mobility and its impact on the dynamics of vector-borne diseases. Human mobility, characterized by the movement of individuals between different geographical locations, plays a crucial role in the spread of these diseases. The proposed model incorporates various factors influencing mobility, such as daily commuting patterns, long-distance travel, and migration, to understand how these movements affect the spatial and temporal distribution of diseases. This model is expected to contribute to design of more effective control strategies to mitigate the impact of these diseases on public health.
	\end{abstract}
\tableofcontents
\listoffigures
% \listoftables
% \listofalgorithms
\lstlistoflistings

\mainmatter

\chapter*{Introducción}\label{chapter:introduction}
\addcontentsline{toc}{chapter}{Introducción}

\chapter{Estado del Arte}\label{chapter:state-of-the-art}
El problema de predicción de la siguiente ubicación de una persona ha encontrado 
aplicación en diversos campos, como sistemas de recomendación (\cite{xue2021mobtcast}), 
redes de sensores (\cite{pirozmand2014human}) y análisis del comportamiento de movilidad~(\cite{wang2021reinforced,xu2022understanding}). 
La definición exacta del problema varía entre los estudios dependiendo de 
los diferentes objetivos y conjuntos de datos empleados. Por ejemplo, 
las aplicaciones de redes sociales basadas en la ubicación (LBSN\footnote{LBSN: Location-Based Social Networks}) se 
centran en predecir el siguiente punto de interés (POI\footnote{POI: Point of Interest}) 
registrado (\cite{wang2022online,xue2021mobtcast}). En contraste, los estudios del comportamiento 
de movilidad buscan comprender la siguiente ubicación de un usuario para 
realizar una actividad~(\cite{solomon2021analyzing}). Aquí nos centramos en los métodos 
propuestos para aplicaciones de movilidad.

\section{Modelos de Markov}
La última década ha presenciado la expansión de estudios centrados en la 
predicción de la siguiente ubicación. La Cadena de Markov y sus variantes 
son probablemente los métodos más empleados para esta tarea~(\cite{luca2021survey}). 
Estos modelos consideran las ubicaciones como estados y construyen una 
matriz de transición que codifica la probabilidad de transición entre 
estados para cada individuo. \cite{ashbrook2002learning} y \cite{gambs2012next} propusieron identificar ubicaciones significativas 
 a partir de datos GPS y construir un modelo de Markov para predecir 
 las transiciones de ubicación. Posteriormente, las variantes del 
 modelo de Markov que consideran movimientos colectivos~(\cite{chen2014nlpm}) e 
 incorporan la importancia de la ubicación~(\cite{huang2017mining}) aumentaron aún 
 más el rendimiento de la predicción. Sin embargo, los modelos basados
en Markov tienen dificultades para representar los complejos patrones
secuenciales en la movilidad humana debido a su supuesto inherente 
de que el estado actual solo depende de los estados de pasos de tiempo
 previamente limitados~(\cite{li2020hierarchical}).

\section{Modelos de Aprendizaje Profundo}
Los avances recientes en el aprendizaje profundo (DL\footnote{DL: Deep Learning}) también han 
promovido su aplicación en la predicción de ubicación. Como método 
de modelado de secuencias ampliamente adoptado, se ha informado que 
los modelos basados en redes neuronales recurrentes (RNN\footnote{RNN: Recurrent Neural Networks}), como es el caso del
modelo de memoria a corto y largo plazo (LSTM\footnote{LSTM: Large Short-Term Memory})~(\cite{solomon2021analyzing}) y la RNN 
espacio-temporal (ST\footnote{ST: Spatial-Temporal})~(\cite{liu2016predicting}), superan a los modelos de Markov
 por un amplio margen en esta tarea. Aún así, los modelos RNN 
 convencionales tienden a subestimar las dependencias a largo 
 plazo cuando aumenta la longitud de la secuencia de entrada. 
 Por lo tanto, los estudios emplearon un mecanismo de atención 
 para capturar dinámicamente las dependencias tanto a corto 
 como a largo plazo~(\cite{feng2018deepmove,li2020hierarchical}). Además, el modelo Transformer, 
 que se basa en el mecanismo de autoatención multicabezal (MHSA\footnote{MHSA: Multi-Head Self-Attentional})~(\cite{vaswani2017attention}), 
 ha comenzado a ganar interés en el campo. En particular, \cite{xue2021mobtcast}
 propusieron MobTcast para considerar 
  varios contextos con una estructura basada en Transformer 
  y se obtuvieron resultados de vanguardia en la predicción de POI 
  para datos LBSN. A pesar de tener un gran potencial para 
  aprender las complejas dependencias espacio-temporales, pocos 
  estudios han aplicado Transformer al problema de predicción 
  de ubicación debido al acercamiento reciente de estos modelos a la problem\'atica en cuesti\'on.

\section{Desaf\'ios del Problema}
Comprender los factores que afectan la elección de la ubicación de 
la actividad es beneficioso para predecir la movilidad de las personas, 
ya que pueden considerarse como conocimiento previo y potencialmente 
guiar el aprendizaje de los modelos DL. En el entorno de la movilidad, 
la elección de ubicaciones se considera una parte integral
del comportamiento de actividad-viaje de las personas y se ha
estudiado dentro del marco basado en la actividad~(\cite{schonfelder2016urban}). 
Los estudios que se centran en analizar el comportamiento de 
viaje a lo largo del tiempo sugieren que se encuentra tanto 
estabilidad como variabilidad en las elecciones de ubicación 
de actividad de las personas. Por ejemplo, \cite{dharmowijoyo2017analysing} mostraron que la variabilidad de las visitas
a ubicaciones es mucho mayor entre los pares fin de 
semana-día laborable que entre los pares día laborable-día 
laborable y fin de semana-fin de semana. Los estudios 
empíricos también demuestran la correlación de diferentes
aspectos del comportamiento de viaje individual. Por ejemplo, \cite{susilo2014repetitions} informaron una alta repetición 
en las combinaciones de ubicación-modo de viaje, lo que sugiere que las 
personas usan el mismo modo de viaje para llegar a sus ubicaciones. 
\cite{hong2022conserved} llegaron a conclusiones similares, donde 
encontraron que solo un subconjunto de todas las combinaciones de 
ubicación-modo es esencial para describir el comportamiento de movilidad. 
Desde esta perspectiva, los aspectos del comportamiento de viaje 
pueden considerarse restricciones para la elección de ubicaciones de 
actividad de las personas.

Un problema similar a la predicción de la siguiente ubicación es la 
formulación del conjunto de opciones de ubicación de un individuo, que 
es un componente crucial en los modelos de microsimulación de 
tráfico~(\cite{mariante2018modeling}). En lugar de predecir la siguiente ubicación exacta, 
el problema se centra en generar un conjunto que contenga todas las 
ubicaciones posibles. Basándose en la teoría de la geografía del tiempo, 
se ha aplicado el análisis de áreas de rutas potenciales para abordar 
el problema, lo que sugiere que el conjunto de opciones está limitado 
por el tiempo de viaje~(\cite{scott2012modeling}), la hora del día~(\cite{yoon2012feasibility}) y el modo 
de viaje disponible~(\cite{neutens2007space}). Sin embargo, este conocimiento empírico 
no se utiliza plenamente en los modelos para predicciones de ubicación.

La mayoría de los estudios existentes solo utilizan secuencias de 
visitas de ubicaciones por los individuos para predecir sus 
próximas ubicaciones, independientemente del contexto de movilidad 
actual (\cite{laha2018real}). Esto se debe a la complejidad y 
diversidad de los datos de contexto (\cite{tedjopurnomo2020survey}). 
Con el desarrollo del aprendizaje automático, los estudios han 
comenzado a explorar métodos para la fusión de datos de contexto y 
emplearlos para impulsar la investigación de la movilidad humana 
(\cite{zheng2018survey,lau2019survey}). Sin embargo, los datos de 
contexto de múltiples fuentes poseen formatos complejos, lo que 
dificulta su fusión en una representación unificada (\cite{liao2018multi}). 
Los marcos de DL sensibles al contexto proporcionan 
una solución para combinar información de contexto de múltiples fuentes 
(\cite{sun2022tcsanet}), pero con mayor frecuencia se centran en el aspecto 
temporal y pasan por alto las interacciones espacio-temporales complejas.\\

Mencionar que la evoluci\'on de la problem\'atica y el auge de los grandes modelos de lenguaje natural (LLM\footnote{LLM: Large Language Models})
han puesto a investigar otras formas de atacar el problema con muy buenos resultados (\cite{wang2024inextlargelanguage}).
De aqu\'i y del tema de la utilizaci\'on del contexto para mejorar la predicci\'on, que se mantenga la necesidad de explorar
m\'as a fondo las influencias de la informaci\'on de contexto del individuo. En particular, se har\'a
enf\'asis en el uso de sentimientos
o emociones que puedan influir en la elecci\'on de la siguiente ubicaci\'on. 
Dichos sentimientos, que en un escenario ideal podr\'ian ser extra\'idos de las redes sociales, 
por falta de datos p\'ublicos se hace casi inviable. Entonces ser\'ia pr\'actico el empleo de los ya 
mencionados modelos del lenguaje natural para inferir un sentimiento 
a partir de los datos que se tienen como lo son la ubicaci\'on y el momento actual del individuo.
Luego, queda hacer uso de un modelo en que 
se tengan en cuenta dichos sentimientos para mejorar la predicci\'on de la siguiente ubicaci\'on.
Se propone entonces utilizar el modelo de Aprendizaje Profundo basado en Transformers y MHSA de \cite{Hong_2023} 
por sus excelentes resultados en esta problem\'atica y su capacidad 
de capturar dependencias.
\chapter{Propuesta}\label{chapter:proposal}
\section{Definición del Problema}

A continuación aparecen un conjunto de términos y nociones 
que ser\'an utilizados en el resto del documento y habr\'a una formulación del problema de 
predicción de la siguiente ubicación. Los datos de movilidad son 
recopilados típicamente a través de dispositivos electrónicos y se 
almacenan como trayectorias espacio-temporales. Cada punto de 
seguimiento en la trayectoria de un usuario contiene un par de 
coordenadas espaciales y una marca de tiempo.\\

\textbf{Definición 1 (Trayectoria GNSS\footnote{GNSS: Global Navigation Satellite System}).} 
Sea $u_i$ un usuario del conjunto de usuarios 
$\mathcal{U} = \{u_1, \dots, u_{|\mathcal{U}|} \}$, una trayectoria 
$T_i = (q_k)_{k=1}^{n_{u_i}}$ es una secuencia ordenada en el 
tiempo compuesta por $n_{u_i}$ puntos de seguimiento visitados por
el usuario $u_i$. Un punto de seguimiento se puede representar 
como una 
tupla $q = \langle p, t \rangle$, donde $p = \langle x, y \rangle$ 
representa las coordenadas espaciales en un sistema de referencia, 
en este caso latitud y longitud, y $t$ es el tiempo de registro.\\

Los puntos de permanencia se detectan a partir de trayectorias GNSS 
sin procesar para identificar áreas donde los usuarios permanecen 
estacionarios durante un período mínimo de tiempo (\cite{li2008mining}). 
Luego, las ubicaciones se forman mediante la agregación espacial 
de puntos de permanencia para caracterizar la semántica del lugar (\cite{hariharan2004project,martin2023trackintel}).
\newpage
\textbf{Definición 2 (Punto de Permanencia).} 
Un punto de permanencia $S = (q_k)_{k=start}^{end}$ es una 
subsecuencia de la trayectoria $T_i$ donde el usuario $u_i$ estuvo 
estacionario desde el punto de seguimiento inicial $q_{start}$ 
hasta el punto de seguimiento final $q_{end}$. Cada punto de 
permanencia $S$ se puede representar como una tupla 
$\langle t, d, g(s) \rangle$, donde $t$ y $d$ representan la 
marca de tiempo de inicio y la duración de la permanencia, 
respectivamente, y $g(s)$ denota la geometría, a menudo 
representada como el centro de sus puntos de seguimiento. 
$S_k$ es usada para denotar el $k$-ésimo punto de permanencia 
en la trayectoria GNSS de un usuario.\\


\textbf{Definición 3 (Ubicación).} Una ubicación $L$ consiste en 
un conjunto de puntos de permanencia espacialmente próximos. Se 
puede representar como una tupla $L = \langle l, g(l) \rangle$, 
donde $l$ es el identificador de la ubicación, 
y $g(l)$ denota la geometría de la ubicación, calculada como la 
envolvente convexa de todos los puntos de permanencia contenidos. 
Por lo tanto, cada ubicación se define como un área. 
$\mathcal{O}_i$ es definida como el conjunto que contiene las 
ubicaciones conocidas para el usuario $u_i$, y 
$\mathcal{O} = \{\mathcal{O}_1, \dots, \mathcal{O}_{|\mathcal{U}|} \}$ 
como el conjunto que contiene todas las ubicaciones.\\


Mediante la generación de ubicaciones cada punto de permanencia 
se enriquece, y si es añadida tambi\'en información de contexto que 
representa un sentimiento del usuario en dicho punto, ll\'amese a este $f$, se tiene
$S = \langle t, d, g(s), f, l, g(l) \rangle$; y entonces la movilidad de 
un usuario se puede representar como una secuencia ordenada en 
el tiempo de $N$ puntos de permanencia visitados $(S_k)_{k=1}^N$.
A continuación, el problema de predicción de la 
siguiente ubicación ser\'ia:\\

\textbf{Problema 1 (Predicción de la Siguiente Ubicación).} 
Considere una secuencia de puntos de permanencia con información 
de contexto $(S_k)_{k=m}^n$ visitada por el usuario $u_i$ en una 
ventana de tiempo desde el paso de tiempo $m$ hasta $n$. El 
objetivo es predecir la ubicación que el mismo usuario visitará 
en el siguiente paso de tiempo, es decir, el identificador de 
ubicación $l_{n+1} \in \mathcal{O}$.\\

La longitud de la ventana temporal determina cuánta información 
histórica se considera en el modelo predictivo. Aquí, se construye 
la secuencia histórica teniendo como base la movilidad realizada en 
los últimos \(D \in \{0, 1, \dots, 14\}\) días.
Por lo tanto, la longitud de la ventana histórica depende del 
usuario \(u_i\) y del paso de tiempo actual \(n\). 
La predicción de la siguiente ubicación se define como un 
problema de predicción de secuencia con longitudes de secuencia 
variables.
\newpage
\section{Metodolog\'ia}

Por sus resultados destacados, proponemos hacer uso de la red neuronal 
que utiliza información de contexto para 
abordar la predicción de la siguiente ubicación de \cite{Hong_2023}
con algunas modificaciones para adaptarla a la problemática. Primero, representamos el contexto 
como los sentimientos de los usuarios en los puntos de permanencia. 
Luego, el modelo utiliza varias capas de \textit{embedding}\footnote{\textit{embedding}: 
técnica de aprendizaje automático que convierte datos de entrada en
representaciones matem\'aticas permitiendo capturar relaciones
sem\'anticas y estructurales entre los datos.} para representar los 
datos heterogéneos de movimiento y contexto. Finalmente, 
se adapta la red MHSA para aprender las dependencias de la 
secuencia histórica e inferir la siguiente ubicación visitada. 
A continuación, se proporciona una descripción 
detallada de cada mo\'dulo.

\subsection{Representación del contexto como sentimientos}
\label{sec:sents}
Para capturar el contexto de los puntos de permanencia utilizamos 
los sentimientos de los usuarios en estos puntos. Sentimientos que son
obtenidos a partir de un LLM capaz de inferirlos a partir
de la ubicación, hora, d\'ia de la semana, as\'i como del punto de inter\'es
visitado (hospital, restaurante, etc). Si bien, al ser un LLM 
no entrenado para este prop\'osito
no es capaz de inferir sentimientos con total precisión
asumimos que pueda ser suficiente para el prop\'osito de este trabajo.

Los sentimientos utilizados
son los siguientes: \textit{miedo}, \textit{hambre}, 
\textit{enfermedad}, \textit{indiferencia}, \textit{cansancio}. Fueron 
escogidos estos y no otros por la motivaci\'on de establecer puentes futuros 
entre los resultados de \cite{Hernandez2023},
y los de este trabajo, en pos de ayudar a la toma de decisiones
en el \'ambito de la salud p\'ublica.

Una vez inferidos por el modelo del lenguaje, los sentimientos pasan a ser 
representados como vectores de \textit{embedding} y son utilizados
para entrenar el modelo de predicción de ubicación; \(e_{f_k}\) representa
el \textit{embedding} del sentimiento \(f\) en el punto de permanencia \(S_k\).\\

\begin{figure}
    \centering
\begin{tikzpicture}[transform shape, scale=0.7,node distance=2cm]

    % Nodes
    \node (useri) [startstop] {\(u_i\)};
    \node (user) [process, below of=useri] {Embedding de usuario};
    \node (sq1) [squares, below of=user] {};
    \node (sq1l) [squares, left of=sq1,xshift=1cm] {};
    \node (sq1r) [squares, right of=sq1,xshift=-1cm] {};
    \node (sq2) [plus, right of=sq1,xshift=0.5cm] {\textbf{+}};
    \node (fcrb) [process, below of=sq2] {FC bloque residual};
    \node (sq3l) [squares, right of=sq2,xshift=-0.5cm] {};
    \node (sq3) [squares, right of=sq3l,xshift=-1cm] {};
    \node (sq3r) [squares, right of=sq3,xshift=-1cm] {};
    \node (sq4) [squares, below of=fcrb] {};
    \node (sq4l) [squares, left of=sq4,xshift=1cm] {};
    \node (sq4r) [squares, right of=sq4,xshift=-1cm] {};
    \node (mhsa) [process, above of=sq3] {MHSA};
    \node (loss) [plus,left of=sq4l,xshift=-1cm] {\(l_{n+1}\)};
    \node (loss2) [plus,below of=sq4,yshift=0.9cm] {\(\hat{l}_{n+1}\)};
    \node (plus) [plus, right of=mhsa,xshift=2cm] {\textbf{+}};
    \node (le) [process,right of=plus,xshift=1.5cm] {Embedding de ubicación};
    \node (de) [process, below of=le] {Embedding de Duraci\'on};
    \node (te) [process, below of=de] {Embedding de Tiempo};
    \node (se) [process, below of=te] {Embedding de Sentimiento};
    \node (l) [startstop, right of = le,xshift=2cm] {\(l_k\)};
    \node (d) [startstop, right of = de,xshift=2cm] {\(d_k\)};
    \node (t) [startstop, right of = te,xshift=2cm] {\(t_k\)};
    \node (f) [startstop, right of = se,xshift=2cm] {\(f_k\)};
    \node (pe) [plus,above of =le] {Codificación posicional};
    \node (sk) [plus,above of = l,yshift=-0.5cm] {\(S_k\)};
    \node[draw, dashed, inner sep=5pt, fit=(l) (d) (t) (f) (sk)] (group) {};
    \node [draw, dashed, inner sep=8pt, fit=(pe) (le) (de) (te) (se) (group) (plus)] (group2) {};
     % Coordinates for the variables


    % arrows
    \draw [arrow] (pe) -| (plus);
    \draw [arrow] (l) -- (le);
    \draw [arrow] (d) -- (de);
    \draw [arrow] (t) -- (te);
    \draw [arrow] (f) -- (se);
    \draw [arrow] (plus) -- (mhsa) node[midway,above] {\(e_{all_k}\)};
    \draw [doublearrow,dashed] (sq4l) -- (loss) node[midway,above] {\(\mathcal{L}\)};
    \draw [arrow] (fcrb) -- (sq4) node[midway, right] {Softmax};
    \draw [arrow] (sq2) -- (fcrb);
    \draw [arrow] (useri) -- (user);
    \draw [arrow] (user) -- (sq1) node[midway, left] {\(e_{u_i}\)};
    \draw [arrow] (mhsa) -- (sq3);
    \draw [arrow] (se) -| (plus) ;
    \draw [arrow] (te) -| (plus) ;
    \draw [arrow] (de) -| (plus) ;
    \draw [arrow] (le) -- (plus) ;
    \end{tikzpicture}
    \caption{Capas de embedding y la red basada en MHSA para 
    la predicción de la próxima ubicación.}
    \label{fig:1}
\end{figure}
\subsection{Generación de \textit{embeddings} espacio-temporales}

Un modelo preciso de predicción de ubicación requiere una selección 
y modelado adecuados de la información de la secuencia histórica. 
Además del identificador de ubicación sin procesar y la hora de 
visita correspondiente que se incluyen a menudo (\cite{li2020hierarchical}), 
consideramos la duración de la actividad y las funciones de uso del 
suelo $g(s),g(l)$ para describir cada punto de permanencia visitado, lo que 
garantiza una representación completa de su contexto desde una 
perspectiva espacio-temporal. Además, la información relacionada 
con el usuario ayuda a descubrir las secuencias recorridas por 
diferentes de ellos y ayuda a la red a aprender patrones de 
movimiento específicos del usuario.

Se utilizan capas de \textit{embedding} para representar características 
del tipo categórico a un vector de valores reales. A diferencia de la 
representación \textit{one-hot}\footnote{\textit{one-hot}: Es una codificación que 
representa variables categóricas como vectores binarios donde cada 
elemento corresponde a una categoría. Solo el elemento que 
representa la categoría activa se establece en 1; todos los demás 
son 0. Una técnica común para manejar datos categóricos en 
aprendizaje automático.}
más clásica, los vectores de \textit{embedding} 
son más compactos y pueden capturar eficazmente la correlación 
latente entre diferentes tipos de características (\cite{xu2022understanding}). 
Estas capas son matrices de parámetros que proporcionan mapeos 
entre la variable original y el vector de \textit{embedding}, optimizadas 
conjuntamente con toda la red. El proceso de \textit{embedding}, as\'i
como la arquitectura del modelo, se muestran en la Figura \ref{fig:1}.

Operacionalmente, dado un punto de permanencia \(S_k\) en la 
secuencia histórica, su identificador de ubicación \(l_k\), 
la hora de llegada \(t_k\) y la duración de la estancia \(d_k\) 
se introducen en sus respectivas capas de \textit{embedding} para 
generar representaciones vectoriales:

\begin{equation}
    e_{l_k} = h_l(l_k; \mathbf{W}_l), 
    e_{t_k} = h_t(t_k; \mathbf{W}_t), 
    e_{d_k} = h_d(d_k; \mathbf{W}_d) \tag{1}
    \label{eq:1}
\end{equation}

donde \(e_{l_k}\), \(e_{t_k}\) y \(e_{d_k}\) son los respectivos 
vectores de \textit{embedding} para \(l_k\), \(t_k\) y \(d_k\). 
En el caso de \(h(\cdot; \cdot)\) 
denota la operación de \textit{embedding} y los términos 
\(\mathbf{W}\) son las matrices de parámetros optimizadas 
durante el entrenamiento. Con \textit{embedding} se convierte por separado los minutos, 
la hora y el día de la semana a partir de la hora de llegada 
\(t_k\) para capturar diferentes niveles de periodicidad en 
las visitas históricas.\\

Finalmente, el vector de \textit{embedding} general 
$e_{all_k}$ para el punto de permanencia $S_k$ se obtiene 
añadiendo sus características espacio-temporales, as\'i como su contexto
basado en sentimientos $e_{f_k}$, junto con una codificación posicional 
$PE$ que codifica la información de secuencia $k$:

\begin{equation}
    e_{all_k}  = e_{l_k} + e_{t_k} + e_{d_k} + e_{f_k} + PE \tag{2}
\end{equation}

El modelo usa la codificación posicional original propuesta por 
\cite{vaswani2017attention} que utiliza funciones seno y coseno. 
La inclusión de la codificación posicional es esencial para 
entrenar una red de autoatención, ya que no asume implícitamente 
el orden secuencial de su entrada (\cite{vaswani2017attention}). Además, 
se representa al usuario $u_i$ del cual se registra la secuencia 
de puntos de permanencia en un vector $e_{u_i}$ con una 
capa de \textit{embedding} de usuario, es decir, 
$e_{u_i} = h_u(u_i; W_u)$. La inclusión de la información 
del usuario asegura que un modelo entrenado con datos de 
población aún pueda distinguir las trayectorias recorridas por 
diferentes usuarios. Como resultado, obtenemos el vector de \textit{embedding} 
general ${e}_{all_k}$ que codifica las características 
espacio-temporales y de contexto, y el vector de 
\textit{embedding} de usuario ${e}_{u_i}$ para la secuencia.


\subsection{Red de autoatención multicabezal}

Una vez se adquieren los vectores de características 
espacio-temporales densos en cada paso de tiempo, debemos extraer 
sus patrones de transición secuencial. Estos patrones históricos 
se capturan utilizando una red basada en MHSA, un mecanismo 
propuesto originalmente dentro de la red transformadora para 
abordar las tareas de traducción de idiomas (\cite{vaswani2017attention}). 
Se adopta una arquitectura similar a las redes GPT\footnote{GPT: Generative Pre-trained Transformer} que solo incluye la parte del decodificador 
del modelo transformador (\cite{radford2018improving}). El decodificador 
consta de una pila de \(L\) bloques idénticos, cada uno con dos 
componentes. El primero es la red de autoatención multicabezal enmascarada 
y el segundo es una red de avance con dos capas lineales, 
separadas por una función de activación ReLU\footnote{ReLU 
(Rectified Linear Unit): se define como 
$f(x) = \max(0, x)$, introduciendo no linealidad y siendo 
computacionalmente eficiente.}. Son agregadas
conexiones residuales, normalización de capa y capas de abandono 
a cada componente para facilitar el aprendizaje. 

La salida del modelo MHSA \(out_n\) se añade al \textit{embedding} del 
usuario \(e_{u_i}\) y juntas se introducen en un bloque residual 
completamente conectado (FC\footnote{FC: Fully Connected}). Finalmente, 
la probabilidad 
predicha de cada ubicación se obtiene mediante una transformación 
\textit{softmax}\footnote{\textit{softmax}: Es una función que convierte 
un vector de valores reales en una distribución de probabilidad.}:

\begin{equation}
P(\hat{l}_{n+1}) = \text{Softmax}(f_{FC}(out_n + e_{u_i}; \mathbf{W}_{FC}))  \tag{3}
\label{ec:3}
\end{equation}

donde \(f_{FC}(\cdot; \cdot)\) representa la operación del 
bloque residual FC. Este bloque consta de capas lineales con 
conexiones residuales, con el objetivo de aprender las dependencias 
entre la información de la secuencia y el usuario para extraer 
las preferencias de movilidad personal. 
\(P(\hat{l}_{n+1}) \in \mathbb{R}^{|\mathcal{O}|}\) contiene la 
probabilidad de que se visiten todas las ubicaciones en el 
siguiente paso de tiempo.

Durante el entrenamiento, con acceso a la siguiente ubicación 
real \(l_{n+1}\), la tarea puede considerarse como un problema de 
clasificación multiclase. Por lo tanto, los parámetros del modelo 
se pueden optimizar utilizando la pérdida de entropía cruzada 
multiclase \(\mathcal{L}\):

\begin{equation}
\mathcal{L} = -\sum_{k=1}^{|\mathcal{O}|} P(l_{n+1})(k) \log(P(\hat{l}_{n+1})(k)) \tag{4}
\label{eq:4}
\end{equation}

donde \(P(\hat{l}_{n+1})(k)\) representa la probabilidad predicha 
de visitar la \(k\)-ésima ubicación y \(P(l_{n+1})(k)\) es la verdad 
representada por \textit{one-hot}, es decir, 
\(P(l_{n+1})(k) = 1\) si la siguiente ubicación real es la 
\(k\)-ésima ubicación, y \(P(l_{n+1})(k) = 0\) en caso contrario.
\chapter{Detalles de Implementación y Experimentos}\label{chapter:implementation}
\section{Datos y preprocesamiento}

Se demuestra la efectividad del m\'etodo propuesto mediante la experimentaci\'on en 
un conjunto de datos de seguimiento GNSS.

\subsection{Conjunto de datos Geolife}
El presente es un conjunto de datos p\'ublico de trayectorias GPS recopilado en el 
proyecto GeoLife de Microsoft Research Asia por 182 usuarios durante 
un período de más de tres años (desde abril de 2007 hasta agosto de 2012) (\cite{zheng2011geolife}). 
En este conjunto se registra una amplia gama de movimientos al aire libre 
de los usuarios, que incluyen no solo rutinas diarias como ir a casa o 
al trabajo, sino también algunas actividades de entretenimiento y deportivas, 
como compras, turismo, cenas, senderismo y ciclismo. 

Una trayectoria GPS de este conjunto de datos está representada por una 
secuencia de puntos con marca de tiempo, cada uno de los cuales contiene 
información de latitud, longitud y altitud. Esta serie de datos 
contiene 17,621 trayectorias con una distancia total de aproximadamente 
1.2 millones de kilómetros y una duración total de más de 48,000 horas. 
Estas trayectorias fueron registradas por diferentes dispositivos GPS y 
teléfonos con GPS, y tienen una variedad de tasas de muestreo. 
El 91 por ciento de las trayectorias se registraron en una representación 
densa, por ejemplo, cada 1-5 segundos o cada 5-10 metros por punto.

Se sigue el marco propuesto 
por \cite{Zheng2010GeoLife} y se clasifica las trayectorias GPS en puntos 
de permanencia y etapas de movimiento utilizando la biblioteca 
Trackintel (\cite{martin2023trackintel}).\\

Las trazas de movimiento de los estudios de seguimiento GNSS se 
preprocesan para la predicci\'on de la siguiente ubicaci\'on. 
Pre-filtramos los conjuntos de datos para considerar solo a los 
usuarios observados por m\'as de 50 d\'ias 
en Geolife, garantizando un tiempo de observaci\'on prolongado. 
Se utiliza la cobertura temporal de seguimiento, que 
cuantifica la proporci\'on de tiempo en que se registran los 
desplazamientos de los usuarios, para evaluar as\'i la calidad del 
seguimiento en la dimensi\'on temporal. Despu\'es de este proceso, 
permanecen 45 usuarios en Geolife.

Las ubicaciones se generan a partir de la secuencia individual 
de puntos de permanencia visitados. Consideramos un punto de permanencia como 
una actividad si su duraci\'on es superior a 25 minutos. Luego, los 
puntos de permanencia de actividad se agrupan espacialmente en ubicaciones 
para considerar visitas al mismo lugar en diferentes momentos. Se utiliza 
la funci\'on proporcionada en Trackintel con los par\'ametros 
$\epsilon = 20$ y $\text{num\_samples} = 2$ para generar las ubicaciones 
del conjunto de datos (\cite{Hong2021Clustering}). La Tabla \ref{tabla:estadisticas} 
muestra estad\'isticas b\'asicas para el conjunto de datos.\\

\begin{table}[h]
    \centering
    \caption{Estadísticas básicas del conjunto de datos de movilidad. Se reportan la media y la desviación estándar entre los usuarios.}
    \label{tabla:estadisticas}
    \begin{tabular}{lc}
        \hline
        & \textbf{Geolife} \\ \hline
        Número de usuarios & 45 \\
        Período de seguimiento (días) & $345 \pm 413$ \\
        \#Puntos de permanencia por usuario & $369 \pm 456$ \\
        \#Puntos de permanencia por usuario por día & $2.4 \pm 1.5$ \\
        \#Ubicaciones por usuario & $77 \pm 108$ \\
        Tamaño de las ubicaciones (m$^2$) & $3606 \pm 12275$ \\
        Cobertura de seguimiento (\%) & $44 \pm 24$ \\ \hline
    \end{tabular}
\end{table}

\subsection{Generaci\'on de los sentimientos}
Para la generación de sentimientos, se utilizó el modelo de Google \textit{Gemini-2.0-Flash-Thinking-Exp} 
como nuestro LLM para inferir los sentimientos a partir de 
los datos disponibles. Este modelo fue seleccionado debido a su capacidad de 
procesamiento y a la disponibilidad de una API de acceso gratuito (\cite{gemini_api_docs}).

\subsubsection{Preprocesamiento de Datos}

Con el objetivo de que el LLM pudiera inferir los sentimientos a partir de la 
información de ubicación, hora y día, se realizó un proceso de transformación 
de estos datos para hacerlos más comprensibles para el modelo. La ubicación, 
originalmente representada en coordenadas de latitud y longitud, fue 
convertida a una dirección en lenguaje natural mediante el uso de la 
técnica de \textit{reverse geocoding}, utilizando la API proporcionada 
por \cite{geocode_maps_co}. La hora fue transformada al 
formato \textit{HH:MM AM/PM}, y el día fue representado como el día de 
la semana correspondiente. Con esta información, se construyó un nuevo 
conjunto de datos en formato CSV, el cual fue utilizado como entrada para el LLM.

\subsubsection{Construcción del Prompt}

El prompt utilizado para guiar al LLM en la inferencia de sentimientos 
fue diseñado a través de un proceso iterativo de prueba y error, con el 
fin de asegurar que las respuestas generadas fueran coherentes y relevantes. 
El prompt utilizado fue el siguiente:
\begin{verbatim}
    f"""You are a sentiment analysis expert.
You will be given a dataset and you will have to create a
situational context for each row of this dataset, with the provided 
information, that it is just the time and location;
from this context your main goal is to identify a sentiment from the 
following list of sentiments: {sentiments}.
You will have to return the sentiment that is most prominent in 
the situational context.
If you are unable to identify the sentiment,
you will have to return '{default_sentiment}'. You will have to 
return only the sentiment and a brief explanation of why you chose 
that sentiment.
The Output should be in a structured CSV with columns: 
index, sentiment, explanation. Provide only CSV-formatted output.
The index of the output must be the same of the input. 
The sentiment must be in lowercase.
The explanation should be between double quotes and 
can't have chinese characters.
You must provide output for all rows in the input.

    Example:
        Input: 17,968,Friday,Friday,10:35 AM,12:32 PM,
        "KFC, Chengfu Road, Wudaokou, Dongsheng, Haidian District, 
        Beijing, 100190, China"
        Output: 17,hunger,"Being at KFC during late morning/noon 
        suggests hunger for lunch."
"""
\end{verbatim}

En este prompt, \textit{sentiments} hace referencia a la lista de sentimientos 
mencionados en la sección \ref{sec:sents}, y \textit{default\_sentiment} corresponde a 
uno de estos sentimientos, específicamente la indiferencia.

\subsubsection{Validación de Respuestas}

Para garantizar la consistencia de las respuestas generadas por el LLM, se 
implementó un proceso de validación que asegura que:
\begin{itemize}
\item El \'indice de la filas de salida coincida con el \'indice de las filas de entrada.

\item Los sentimientos devueltos pertenezcan al conjunto de sentimientos predefinidos.

\item El formato de salida cumpla con la estructura especificada: 
\textit{index, sentiment, explanation}.
\end{itemize}
En caso de que alguna de estas condiciones no se cumpliera, se realizaba una 
nueva petición al LLM hasta obtener una respuesta válida.

\subsubsection{Conversión de Sentimientos a Valores Numéricos}

Finalmente, los sentimientos inferidos por el LLM fueron convertidos a 
valores numéricos en el rango de 0 a 4, inclusive. Estos valores fueron 
incorporados al conjunto de datos utilizado para el entrenamiento del 
modelo principal.

\section{Entrenamiento del modelo}

Se divide el conjunto de datos, con la información de los sentimientos
a\~nadida, en conjuntos de
entrenamiento, validaci\'on y prueba sin superposici\'on, con una 
proporci\'on de 6:2:2 basada en el tiempo. Para cada usuario, 
las secuencias de puntos de permanencia correspondientes al 
primer 60\% de los días de seguimiento se emplean para el 
entrenamiento del modelo, mientras que el 20\% final se 
reserva para la fase de prueba. Los parámetros de la red 
de predicción se ajustan utilizando el conjunto de 
entrenamiento, y el conjunto de validación se emplea para 
monitorear la pérdida del modelo. 
Se efect\'ua \textit{grid search}\footnote{\textit{grid search}: 
T\'ecnica tradicional de optimización de 
hiperparámetros, que no es más que una búsqueda exhaustiva a través de 
un subconjunto especificado manualmente del espacio de hiperparámetros 
de un algoritmo de aprendizaje. Un algoritmo de \textit{grid search} debe 
guiarse por alguna métrica de rendimiento, normalmente medida por validación 
cruzada en el conjunto de entrenamiento o evaluación en un conjunto de validación de espera.}
sobre los hiperpar\'ametros en el conjunto de validaci\'on. Finalmente, 
se eval\'ua y se reporta el 
desempe\~no del modelo utilizando el conjunto de prueba.

Durante el entrenamiento, se minimiza la Ecuaci\'on (\ref{eq:4}) con el optimizador 
Adam sobre lotes de muestras de datos de entrenamiento, con una tasa de 
aprendizaje inicial de $1e^{-3}$ y una penalizaci\'on L2 de $1e^{-6}$. 
Se adopta una estrategia de parada temprana para detener el aprendizaje si 
la p\'erdida de validaci\'on deja de disminuir durante 3 \'epocas. Luego, 
la tasa de aprendizaje se multiplica por 0.1 y el entrenamiento se 
reanuda desde el modelo con la menor p\'erdida de validaci\'on. Este proceso 
de parada temprana se repite 3 veces. Adem\'as, se implementa un 
calentamiento de la tasa de aprendizaje durante 2 \'epocas y una deca\'ida 
lineal de 0.98 por \'epoca posteriormente (\cite{vaswani2017attention}).

\section{Modelos de predicci\'on de referencia}

Se compara el rendimiento de este modelo con el m\'etodo 
clásico de predicci\'on de ubicaci\'on basado en Markov y con el modelo 
de predicci\'on de ubicaci\'on sobre el que est\'a construido 
el modelo propuesto.

\begin{itemize}
    \item \textbf{Markov}. Los modelos cl\'asicos de predicci\'on de 
    ubicaci\'on asumen la propiedad de Markov en las visitas a ubicaciones 
    individuales (\cite{ashbrook2002learning}). Implementamos la Cadena de Markov de 
    primer orden (1-MMC) (\cite{gambs2012next}), ya que aumentar el orden no 
    mejora el rendimiento de la predicci\'on.
    \item \textbf{MHSA Transformer}. El modelo MHSA Transformer es el 
    modelo sobre el que se construye nuestro modelo propuesto (\cite{Hong_2023}). 
    La diferencia entre estos 
    radica en que el primero no utiliza informaci\'on de contexto adicional
    para realizar la predicci\'on de 
    ubicaci\'on, como es el caso de los sentimientos.
\end{itemize}

\section{M\'etricas de Evaluaci\'on}

Se utilizan las siguientes m\'etricas para cuantificar el rendimiento de 
los modelos implementados:

\begin{itemize}
    \item \textbf{Exactitud (Accuracy).} Mide la correcci\'on de la 
    ubicaci\'on predicha en comparaci\'on con la ubicaci\'on real visitada 
    a continuaci\'on. Pr\'acticamente, se ordena el vector de probabilidades 
    de ubicaci\'on $P (\hat{l}_{n+1})$, obtenido de la Ecuaci\'on (\ref{ec:3}), en orden 
    descendente y se verifica si la ubicaci\'on real aparece entre las 
    k mejores predicciones, Acc@k mide la proporci\'on de veces que esto 
    es cierto en el conjunto de prueba. En la literatura sobre predicci\'on 
    de ubicaciones, esta m\'etrica tambi\'en se conoce como Recall@k 
    o Hit Ratio@k. Se reportan Acc@1, Acc@5 y Acc@10 para permitir 
    comparaciones con otros trabajos.

    \item \textbf{Puntaje F1 (F1).} Las visitas individuales a ubicaciones 
    son altamente desbalanceadas, con ubicaciones espec\'ificas ocurriendo 
    con mayor frecuencia en la rutina diaria que otras. Utilizamos la 
    puntuaci\'on F1 ponderada por el n\'umero de visitas para enfatizar 
    el rendimiento del modelo en las ubicaciones m\'as importantes.

    \item \textbf{Rango rec\'iproco medio (MRR).} Calcula el promedio del 
    rec\'iproco del rango en el que se recuper\'o la primera entrada 
    relevante en el vector de predicci\'on:
    \begin{equation}
        MRR = \frac{1}{N} \sum_{i=1}^{N} \frac{1}{\text{rank}_i} \tag{5}
    \end{equation}
    donde $N$ denota el n\'umero de muestras de prueba y $\text{rank}_i$ 
    es el rango de la ubicaci\'on real en $P (\hat{l}_{n+1})$ para 
    la $i$-\'esima muestra de prueba.

    \item \textbf{Ganancia acumulativa con descuento normalizada (NDCG).} 
    Mide la calidad del vector de predicci\'on por la relaci\'on entre la 
    ganancia acumulativa con descuento (DCG) y la ganancia acumulativa con 
    descuento ideal (IDCG):
    \begin{equation}
        NDCG = \frac{1}{N} \sum_{i=1}^{N} \frac{DCG_i}{IDCG_i}, 
        \quad \text{donde} \quad DCG_i = \sum_{j=1}^{|\mathcal{O}|} \frac{r_j}{\log_2(j + 1)} \tag{6}
    \end{equation}
    donde $r_j$ denota el valor de relevancia en la posici\'on $j$. 
    En el contexto de la predicci\'on de ubicaciones, $r_j$ es binario, 
    es decir, $r_j \in \{0, 1\}$, y $r_j = 1$ si y solo si el $j$-\'esimo 
    elemento en el ordenado $P (\hat{l}_{n+1})$ corresponde a la 
    ubicaci\'on real siguiente. NDCG@k mide la relevancia de los 
    resultados hasta la posici\'on k en el ranking. En la 
    evaluaci\'on, se reporta NDCG@10.
\end{itemize}
\newpage
\section{Resultados}


\subsection{Resultados de Desempe\~no}

\begin{table}[h]
    \centering
    \caption{Resultados de evaluación del rendimiento para la predicción de la próxima ubicación 
    sobre el conjunto de datos Geolife. Se reporta el promedio y la desviación estándar 
    en 12 ejecuciones diferentes.}
    \label{tab:2}
    \resizebox{1\textwidth}{!}{
    \begin{tabular}{lccccccc}
        \toprule
        Método & Acc@1 & Acc@5 & Acc@10 & F1 & MRR & NDCG@10 \\
        \midrule
         1-MMC & 24.1 & 38.1 & 39.5 & 22.7 & 30.5 & 32.7 \\
         MHSA Transformer & 29.4 ± 0.8 & 53.6 ± 1.4 & 57.8 ± 1.1 & 19.7 ± 1.5 & 40.2 ± 0.8 & 44.2 ± 0.9 \\
         Nuestro modelo & 29.6 ± 0.8 & 53.8 ± 1.5 & 57.8 ± 1.5 & 20.2 ± 1.0 & 40.5 ± 0.7 & 44.4 ± 0.8 \\
        \bottomrule
    \end{tabular}
    }
\end{table}




Se presenta primero el desempe\~no de predicci\'on para todos los 
m\'etodos considerados en la Tabla \ref{tab:2}. Para cada modelo basado en 
aprendizaje, se entrena el modelo doce veces con diferentes 
inicializaciones aleatorias de los par\'ametros y reportamos 
la media y la desviaci\'on est\'andar de los indicadores de 
desempe\~no respectivos. Se utiliza la prueba U de Mann-Whitney para 
verificar si las diferencias de desempe\~no entre los distintos modelos 
son significativas. Los modelos de aprendizaje profundo (DL) se 
entrenan en los datos de toda la poblaci\'on, utilizando identificadores 
de usuario para distinguir las secuencias registradas de diferentes usuarios. 
Se introducen secuencias hist\'oricas de los \'ultimos $D = 7$ d\'ias en 
todos los modelos DL para garantizar su comparabilidad.

Se reporta que MHSA Transformer supera al m\'etodo 1-MMC en todos los indicadores 
excepto en la puntuaci\'on F1. La brecha de desempe\~no es grande en 
Acc@5, Acc@10, MRR y NDCG@10, lo que implica que MHSA Transformer puede identificar 
mejor las preferencias del usuario al considerar el conocimiento 
colectivo de movilidad. La puntuaci\'on F1 relativamente alta del 
m\'etodo 1-MMC sugiere que este m\'etodo es pr\'actico si el objetivo es 
la predicci\'on de ubicaciones esenciales. Sin embargo, sus 
desempe\~nos siguen siendo significativamente inferiores a los modelos 
basados en DL en el conjunto de datos considerado. Esta diferencia 
enfatiza la importancia de considerar dependencias a largo plazo y 
contextos espaciotemporales en la tarea de predicci\'on.

\begin{table}[h]
    \centering
    \caption{Resultados de la prueba U de Mann–Whitney para la comparación de MHSA Transformer y el modelo propuesto.}
    \label{tab:3}
    \begin{tabular}{lc}
        \toprule
        Métrica & Valor p \\
        \midrule
        NDCG@10 & 0.750832 \\
        F1      & 0.157213 \\
        Acc@1   & 0.772734 \\
        Acc@5   & 0.862312 \\
        Acc@10  & 0.685977 \\
        MRR     & 0.506721 \\
        \bottomrule
    \end{tabular}
\end{table}
\newpage
El modelo propuesto, que utiliza como contexto los 
sentimientos, obtiene los mejores resultados en todos los 
indicadores, presentando ligeramente mejores promedios en 
comparación con el MHSA Transformer. Sin embargo, 
la comparación de los resultados entre ambos enfoques 
revela diferencias muy pequeñas, con variaciones en el 
orden de 0.2 a 0.5 puntos en promedio, que se encuentran 
dentro del rango de las desviaciones estándar reportadas. 
Además, la aplicación de la prueba de Mann–Whitney U a 
cada métrica resultó en valores p superiores a 0.05 (Ver Tabla \ref{tab:3}), lo que indica que no 
existen diferencias estadísticamente significativas entre 
los métodos. Estas mínimas discrepancias en el desempeño 
sugieren que las pequeñas mejoras observadas en nuestro 
modelo podrían atribuirse a la variabilidad inherente en 
las 12 ejecuciones experimentales, en lugar de a una 
ventaja real del uso del contexto de sentimientos. 
En conjunto, estos resultados apoyan la conclusión de que, 
aunque el enfoque de incorporar sentimientos como contexto 
logra obtener promedios ligeramente superiores en los 
indicadores evaluados, ambos métodos presentan un 
rendimiento prácticamente similar en la predicción de 
la próxima ubicación sobre el conjunto de datos Geolife, 
sin evidencias concluyentes de que uno supere 
significativamente al otro.

\subsection{Influencia del contexto de sentimientos}
Aunque los resultados experimentales muestran que la 
incorporación del contexto de sentimientos en nuestro modelo 
no genera mejoras estadísticamente significativas en comparación 
con el MHSA Transformer, es importante analizar en detalle cómo 
este contexto influye en la predicción de la próxima ubicación. 

Uno de los factores que pueden haber afectado
 en el desempeño de nuestro modelo 
es la manera en que se 
incorporó el contexto de sentimientos. En este caso, 
los sentimientos fueron inferidos mediante un LLM 
utilizando únicamente información como la hora, el lugar y 
el día de la semana. Dado que el LLM no tenía acceso a 
información subjetiva o explícita de los usuarios, 
las etiquetas de sentimientos generadas no  
capturan de manera precisa el estado real de 
los individuos en cada momento.

Además, debido a la aleatoriedad inherente en la 
generación de texto por parte de los modelos de 
lenguaje, es posible que la asignación de sentimientos no 
haya sido consistente a los datos de los que se infiere, 
lo que introduce una fuente 
adicional de ruido en los datos. Esta incertidumbre 
en la inferencia del contexto de estado de los individuos 
puede haber reducido 
el impacto del componente de sentimientos en la tarea de 
predicción, haciendo que las diferencias con el modelo base 
no sean lo suficientemente pronunciadas para alcanzar 
significancia estadística.

Otro aspecto a considerar es que, aunque el contexto de 
sentimientos puede ser útil en ciertos escenarios, su 
relevancia en la predicción de la próxima ubicación podría 
depender de la naturaleza de los datos de movilidad. 
Si los patrones de movimiento de los usuarios están 
mayormente determinados por factores rutinarios o 
estructurales (por ejemplo, trabajo, estudios, 
transporte público), el efecto de los sentimientos en la 
predicción podría ser marginal. Para evaluar esto, futuras 
investigaciones podrían explorar la incorporación de fuentes 
de datos más ricas, como registros de actividad en redes 
sociales o encuestas directas, para mejorar la calidad de 
la inferencia emocional y analizar si una mayor precisión 
en la estimación del estado de los individuos impacta 
significativamente en la predicción de movilidad.

En conclusión, aunque la integración del contexto de 
sentimientos en nuestro modelo no mostró mejoras 
significativas respecto al MHSA Transformer, esto no 
implica necesariamente que la información de este tipo sea 
irrelevante. Más bien, su impacto puede depender de la 
calidad y fiabilidad de los sentimientos inferidos, 
así como de la influencia real que estos tengan en 
los patrones de movilidad de los usuarios.

\subsection{Impacto de las longitudes de entrada históricas}
A continuación interesa conocer cuánta información histórica 
debe considerarse para que la red DL logre el rendimiento deseado. 
Se identifica el tiempo de visita de cada registro histórico con 
referencia a la predicción actual y alteramos la longitud de la 
secuencia de entrada controlando el número de días $D$ a considerar 
en el pasado. Se puede 
observar una tendencia general decreciente cuando aumenta el 
número de días históricos considerados, lo que significa que 
incluir secuencias más largas, y por lo tanto más información, 
no necesariamente conduce a un mejor rendimiento del modelo. 
Además, observamos dos picos en la tendencia de Acc@1 
correspondientes a visitas a puntos de estancia en los 7 y 14 días 
anteriores. Los resultados de las pruebas de Mann–Whitney U 
muestran que el Acc@1 obtenido de los últimos 7 días no es 
significativamente diferente de considerar 1 día en el 
pasado, pero sí es significativamente 
diferente al obtenido de todas las demás longitudes de entrada. 
Por lo tanto, concluimos 
que el modelo propuesto, as\'i como el MHSA Transformer logran el mejor rendimiento 
al considerar la movilidad realizada en los últimos 7 días. 
Además, los picos de rendimiento sugieren que las huellas de 
movilidad de una o dos semanas atrás llevan información adicional 
que es beneficiosa para predecir la visita al lugar del día actual.

\backmatter

\begin{conclusions}
    En este trabajo, se analiza el desempeño de diferentes modelos para 
    la predicción de la próxima ubicación de usuarios en el conjunto de 
    datos Geolife. Se comparó el modelo basado en MHSA Transformer con 
    nuestro modelo propuesto, el cual incorpora el contexto de sentimientos 
    inferidos mediante un modelo de lenguaje.

Los resultados muestran que ambos modelos basados en aprendizaje profundo 
superan significativamente al método de primer orden 1-MMC en casi todos 
los indicadores de desempeño, resaltando la importancia de modelar 
dependencias temporales a largo plazo en tareas de predicción de 
movilidad. Sin embargo, la inclusión del contexto de sentimientos en 
nuestro modelo no resultó en mejoras estadísticamente significativas en 
comparación con el MHSA Transformer, según la prueba U de Mann-Whitney.

Se identificaron dos posibles razones para esta falta de significancia: 
(i) la inferencia del contexto de sentimientos basada en información 
limitada (hora, lugar y día de la semana) pudo no haber capturado con 
precisión el estado emocional real de los usuarios, introduciendo ruido 
en los datos, y (ii) la influencia de los sentimientos en la movilidad
puede ser marginal cuando los patrones de movimiento están fuertemente 
determinados por factores estructurales, como el trabajo o la rutina diaria.

Por otro lado, el análisis del impacto de la longitud de la secuencia 
histórica reveló que la mejor configuración para el modelo propuesto 
se obtiene al considerar las secuencias de los últimos 7 días. 
Además, los picos en el desempeño observados en los días 7 y 14 sugieren 
que los patrones de movilidad pueden tener periodicidades semanales 
relevantes para la predicción.

En conclusión, si bien el contexto de sentimientos no proporcionó mejoras 
significativas en la predicción de movilidad en este estudio, su 
potencial utilidad podría depender de la calidad de la inferencia 
emocional y del tipo de movilidad analizada.

\end{conclusions}

\begin{recomendations}
    A partir de los resultados, se proponen las siguientes 
    recomendaciones para futuras investigaciones y aplicaciones:

\begin{itemize}

    \item \textbf{Evaluación rigurosa del LLM:} Si se emplea un LLM para 
    la generación de etiquetas de sentimientos, es fundamental realizar 
    una evaluación exhaustiva de su desempeño. Esto incluye verificar 
    la coherencia y estabilidad de las inferencias, comparar los 
    resultados con anotaciones humanas y analizar posibles sesgos o 
    errores sistemáticos que puedan afectar la calidad de los datos.
    
    \item \textbf{Uso de técnicas clásicas de recuperación de información:} 
    En lugar de utilizar un 
    LLM, una alternativa viable es aplicar enfoques tradicionales de 
    recuperación de información, como la búsqueda de patrones en 
    bases de datos preexistentes de estados de la persona, el uso de 
    diccionarios léxicos de sentimientos o modelos supervisados de 
    clasificación de texto. 
    Estos métodos pueden ofrecer mayor control y transparencia en 
    la inferencia de sentimientos, reduciendo la variabilidad 
    introducida por la generación de texto de los LLM.

   

    \item \textbf{Mejora en la inferencia del contexto de sentimientos:} 
    Para capturar mejor el estado emocional de los usuarios, se recomienda 
    el uso de fuentes de datos adicionales, como interacciones en redes 
    sociales, encuestas directas o datos biométricos, que puedan 
    proporcionar una inferencia más precisa y menos ruidosa de los 
    sentimientos.
    
    \item \textbf{Análisis del impacto del contexto de estado en 
    diferentes escenarios:} Se sugiere evaluar el impacto del contexto 
    de sentimientos en distintos conjuntos de datos de movilidad, 
    especialmente en entornos donde las decisiones de movimiento pueden 
    estar más influenciadas por factores emocionales.
    
    \item \textbf{Experimentación y visualización de resultados:}  
    Se recomienda incluir más experimentación para validar el desempeño 
    de los modelos, así como el uso de gráficos e histogramas que 
    permitan ilustrar mejor las comparaciones entre los distintos enfoques,  
    facilitando as\'i la interpretabilidad de los resultados.
\end{itemize}

En general, estos puntos pueden contribuir a mejorar la precisión de 
los modelos de predicción de movilidad y a profundizar en la 
comprensión del papel de los factores de sentimientos en los patrones 
de movimiento de los usuarios.

\end{recomendations}

\printbibliography[heading=bibintoc]


\end{document}